% Options for packages loaded elsewhere
\PassOptionsToPackage{unicode}{hyperref}
\PassOptionsToPackage{hyphens}{url}
%
\documentclass[
  12pt,
]{article}
\usepackage{amsmath,amssymb}
\usepackage{iftex}
\ifPDFTeX
  \usepackage[T1]{fontenc}
  \usepackage[utf8]{inputenc}
  \usepackage{textcomp} % provide euro and other symbols
\else % if luatex or xetex
  \usepackage{unicode-math} % this also loads fontspec
  \defaultfontfeatures{Scale=MatchLowercase}
  \defaultfontfeatures[\rmfamily]{Ligatures=TeX,Scale=1}
\fi
\usepackage{lmodern}
\ifPDFTeX\else
  % xetex/luatex font selection
\fi
% Use upquote if available, for straight quotes in verbatim environments
\IfFileExists{upquote.sty}{\usepackage{upquote}}{}
\IfFileExists{microtype.sty}{% use microtype if available
  \usepackage[]{microtype}
  \UseMicrotypeSet[protrusion]{basicmath} % disable protrusion for tt fonts
}{}
\makeatletter
\@ifundefined{KOMAClassName}{% if non-KOMA class
  \IfFileExists{parskip.sty}{%
    \usepackage{parskip}
  }{% else
    \setlength{\parindent}{0pt}
    \setlength{\parskip}{6pt plus 2pt minus 1pt}}
}{% if KOMA class
  \KOMAoptions{parskip=half}}
\makeatother
\usepackage{xcolor}
\usepackage[margin=1in]{geometry}
\usepackage{graphicx}
\makeatletter
\def\maxwidth{\ifdim\Gin@nat@width>\linewidth\linewidth\else\Gin@nat@width\fi}
\def\maxheight{\ifdim\Gin@nat@height>\textheight\textheight\else\Gin@nat@height\fi}
\makeatother
% Scale images if necessary, so that they will not overflow the page
% margins by default, and it is still possible to overwrite the defaults
% using explicit options in \includegraphics[width, height, ...]{}
\setkeys{Gin}{width=\maxwidth,height=\maxheight,keepaspectratio}
% Set default figure placement to htbp
\makeatletter
\def\fps@figure{htbp}
\makeatother
\setlength{\emergencystretch}{3em} % prevent overfull lines
\providecommand{\tightlist}{%
  \setlength{\itemsep}{0pt}\setlength{\parskip}{0pt}}
\setcounter{secnumdepth}{-\maxdimen} % remove section numbering
\usepackage{caption}
\usepackage{bbm}
\usepackage{booktabs}
\usepackage{longtable}
\usepackage{array}
\usepackage{multirow}
\usepackage{wrapfig}
\usepackage{float}
\usepackage{colortbl}
\usepackage{pdflscape}
\usepackage{tabu}
\usepackage{threeparttable}
\usepackage{threeparttablex}
\usepackage[normalem]{ulem}
\usepackage{makecell}
\usepackage{xcolor}
\ifLuaTeX
  \usepackage{selnolig}  % disable illegal ligatures
\fi
\IfFileExists{bookmark.sty}{\usepackage{bookmark}}{\usepackage{hyperref}}
\IfFileExists{xurl.sty}{\usepackage{xurl}}{} % add URL line breaks if available
\urlstyle{same}
\hypersetup{
  pdfauthor={Guilherme Jardim - 2512502},
  hidelinks,
  pdfcreator={LaTeX via pandoc}}

\title{Segunda Lista de Exercícios:\\
Modelos de Entrada}
\author{Guilherme Jardim - 2512502}
\date{}

\begin{document}
\maketitle

\captionsetup[table]{labelformat=empty}

\hypertarget{questuxe3o-1}{%
\section{Questão 1}\label{questuxe3o-1}}

\textbf{(a)} Para investigar as previsões teóricas, vamos montar um
gráfico ilustrando a distribuição da população para cada número de
empresas na cidade, com um \emph{box plot}.\\
\includegraphics{PS2_2112502-1-_files/figure-latex/unnamed-chunk-2-1.pdf}

É possível observar que de fato há um relação positiva entre o número de
revendedores e a população na cidade, com a distribuição da população se
concentrando em valores maiores conforme aumentamos o número de
revendedores. Além disso, é possível observar que realmente há uma maior
dispersão na população conforme o número de revendedores aumenta, o que
sugere a presença de aumento da variância nos thresholds conforme \(n\)
aumenta. Entretanto, apenas pelo gráfico não é claro que os thresholds
\(S_n\) crescem mais que proporcionalmente, vamos tentar investigar isso
na tabela abaixo.

\begin{verbatim}
## Warning: Returning more (or less) than 1 row per `summarise()` group was deprecated in
## dplyr 1.1.0.
## i Please use `reframe()` instead.
## i When switching from `summarise()` to `reframe()`, remember that `reframe()`
##   always returns an ungrouped data frame and adjust accordingly.
## Call `lifecycle::last_lifecycle_warnings()` to see where this warning was
## generated.
\end{verbatim}

\begin{table}[H]

\caption{\label{tab:unnamed-chunk-3}Distribuição da População pelo Número de Revendedores na Cidade}
\centering
\begin{tabular}[t]{lrrrrrr}
\toprule
Percentil & 0 & 1 & 2 & 3 & 4 & 5+\\
\midrule
0.1 & 0.39 & 0.65 & 1.03 & 1.20 & 1.99 & 3.36\\
0.2 & 0.50 & 0.77 & 1.43 & 1.73 & 2.23 & 4.21\\
0.3 & 0.63 & 0.91 & 1.64 & 1.90 & 2.82 & 5.28\\
0.4 & 0.74 & 1.05 & 1.79 & 2.11 & 3.93 & 5.86\\
0.5 & 0.92 & 1.17 & 2.22 & 2.59 & 4.65 & 6.51\\
\addlinespace
0.6 & 1.04 & 1.35 & 2.69 & 3.07 & 5.92 & 7.56\\
0.7 & 1.19 & 1.67 & 3.16 & 3.36 & 6.82 & 9.75\\
0.8 & 1.31 & 2.44 & 3.79 & 3.78 & 7.00 & 12.07\\
0.9 & 1.68 & 3.40 & 4.47 & 4.84 & 9.10 & 18.14\\
Média & 1.09 & 1.62 & 2.64 & 2.80 & 5.08 & 9.71\\
\bottomrule
\end{tabular}
\end{table}

A partir da tabela, vemos que ao sairmos de um revendedor para dois a
população aproximadamente dobra para alguns dos quantis mas o aumento
não é mais do que proporcional para nenhum deles, sugerindo que não é
visível no dado bruto que os thresholds \(S_n\) crescem mais do que
proporcionalmente de \(n=1\) para \(n=2\). Quando observamos \(n=3\),
essa relação é ainda menos clara, com muitos dos quantis de população
estando próximos dos condicionais a \(n=2\), não dando uma noção clara
do que ocorre de \(s_2\) para \(s_3\). Para \(n=4\), já parece haver uma
diferença maior em relação aos valores de \(n\) menores na distribuição
da população, entretanto, o número de cidades com 4 revendedores é bem
menor do que das categorias anteriores e as comparações já passam a ser
menos convincentes. A tabela abaixo mostra a quantidade de cidades
conforme o número de revendedores.

\begin{table}[H]

\caption{\label{tab:unnamed-chunk-4}Número de Cidades Observadas pelo Número de Revendedores na Cidade}
\centering
\begin{tabular}[t]{lr}
\toprule
Revendedores & Número de Cidades\\
\midrule
0 & 45\\
1 & 39\\
2 & 39\\
3 & 24\\
4 & 13\\
\addlinespace
5+ & 42\\
\bottomrule
\end{tabular}
\end{table}

Para 5 ou mais revendedores, a diferença na distribuição de população é
bem clara, mas pode estar sendo gerada pela agregação de números
diferentes de revendedores em um mesmo grupo, o que é mostrado no
gráfico abaixo. Uma anotação válida é que os grupos com \(n \geq 5\) se
tornam menos numerosos, o que torna nossas comparações novamente menos
convincentes, o que pode argumentar a favor da decisão do artigo
original de unificar o grupo com 5 ou mais firmas.

\includegraphics{PS2_2112502-1-_files/figure-latex/unnamed-chunk-5-1.pdf}

Um ponto relevante a ser levantado é que o \(S\) utilizado no modelo não
se limita à população. Como os outros preditores utilizados em Bresnahan
and Reiss (1991) não estão sendo levados em conta, não é tão direto
afirmar que os \(s_n\) não crescem com \(n\).

\textbf{(b)} Para isso, precisamos montar a função de verossimilhança a
ser maximizada. A função de lucro da firma em um mercado com \(N\)
firmas é dada por: \[
\Pi_{N}=S(\mathbf{Y}, \lambda) V_{N}(\mathbf{Z}, \mathbf{W}, \alpha, \beta)-F_{N}(\mathbf{W}, \gamma)+\epsilon
\]

Onde \(\epsilon\) tem distribuição normal e é independente entre
mercados e das variáveis observáveis. \[
\begin{aligned}
S(\mathbf{Y}, \lambda)=& \text { town population }+\lambda_{1} \text { nearby population } \\
&+\lambda_{2} \text { positive growth }+\lambda_{3} \text { negative growth } \\
&+\lambda_{4} \text { commuters out of the county. }
\end{aligned}
\]

\[
V_{N}=\alpha_{1}+\mathbf{X} \beta-\sum_{n=2}^{N} \alpha_{n}
\]

\[
F_{N}=\gamma_{1}+\gamma_{L} W_{L}+\sum_{n=2}^{N} \gamma_{n}
\]

A probabilidade de observar mercados sem firmas é dada por: \[
\operatorname{Pr}\left(\Pi_{1}<0\right)=1-\Phi\left(\bar{\Pi}_{1}\right)
\]

A probabilidade de observar mercados com \(N\) entre 2 e 4 firmas é dada
por: \[
\operatorname{Pr}\left(\Pi_{N} \geq 0 \text { e } \Pi_{N+1}<0\right)=\Phi\left(\bar{\Pi}_{N}\right)-\Phi\left(\bar{\Pi}_{N+1}\right)
\]

E a probabilidade residual de observar mercados com 5 ou mais firmas é:
\[
\operatorname{Pr}\left(\Pi_{5} \geq 0\right)=\Phi\left(\bar{\Pi}_{5}\right)
\]

Esse modelo é análogo ao Probit Ordenado, então basta maximizar a
verossimilhança dele. Os resultados da replicação são mostrados na
tabela abaixo.

\begin{table}[H]

\caption{\label{tab:unnamed-chunk-7}Tabela 4 - Bresnahan and Reiss (1991)}
\centering
\begin{tabular}[t]{lrr}
\toprule
  & Estimativa & Erro-padrão assintótico\\
\midrule
$\lambda_1$ & -0.53 & 0.40\\
$\lambda_2$ & 2.25 & 0.97\\
$\lambda_3$ & 0.34 & 0.61\\
$\lambda_4$ & 0.23 & 0.41\\
$\beta_2$ & -0.49 & 0.63\\
$\beta_3$ & -0.03 & 0.03\\
$\beta_4$ & 0.00 & 0.06\\
$\beta_7$ & -0.02 & 0.08\\
$\alpha_1$ & 0.86 & 0.46\\
$\alpha_2$ & 0.03 & 0.12\\
$\alpha_3$ & 0.15 & 0.09\\
$\alpha_4$ &  & \\
$\alpha_5$ & 0.08 & 0.05\\
$\gamma_1$ & 0.53 & 0.22\\
$\gamma_2$ & 0.76 & 0.19\\
$\gamma_3$ & 0.46 & 0.20\\
$\gamma_4$ & 0.60 & 0.11\\
$\gamma_5$ & 0.12 & 0.17\\
$\gamma_L$ & -0.74 & 0.40\\
Log likelihood & -263.09 & \\
\bottomrule
\end{tabular}
\end{table}

Os erros-padrão não estão exatamente iguais mas, como a estimação
depende de métodos numéricos, é esperado que haja uma diferença. Ainda
assim, os parâmetros estimados estão iguais aos apresentados por
Bresnahan and Reiss (1991).

\textbf{(c)} Estimando novamente sem a restrição de \(\alpha_4 = 0\)
como no artigo original, temos o resultado na tabela abaixo.

\begin{table}[H]

\caption{\label{tab:unnamed-chunk-9}Tabela 4 - Bresnahan and Reiss (1991) com $\alpha_4$}
\centering
\begin{tabular}[t]{lrr}
\toprule
  & Estimativa & Erro-padrão assintótico\\
\midrule
$\lambda_1$ & -0.46 & 0.43\\
$\lambda_2$ & 2.29 & 1.01\\
$\lambda_3$ & 0.48 & 0.67\\
$\lambda_4$ & 0.20 & 0.44\\
$\beta_2$ & -0.42 & 0.62\\
$\beta_3$ & -0.03 & 0.03\\
$\beta_4$ & 0.01 & 0.06\\
$\beta_7$ & -0.02 & 0.08\\
$\alpha_1$ & 0.82 & 0.46\\
$\alpha_2$ & 0.04 & 0.11\\
$\alpha_3$ & 0.17 & 0.09\\
$\alpha_4$ & -0.06 & 0.07\\
$\alpha_5$ & 0.09 & 0.05\\
$\gamma_1$ & 0.50 & 0.22\\
$\gamma_2$ & 0.74 & 0.18\\
$\gamma_3$ & 0.42 & 0.20\\
$\gamma_4$ & 0.76 & 0.24\\
$\gamma_5$ & 0.11 & 0.17\\
$\gamma_L$ & -0.71 & 0.40\\
Log likelihood & -262.78 & \\
\bottomrule
\end{tabular}
\end{table}

A restrição imposta ao parâmetro \(\alpha_4\) não parece ter grandes
impactos na estimação dos demais coeficientes e o parâmetro estimado
para \(\alpha_4\) não é estatisticamente diferente de zero, o que
argumenta em favor do artigo original. Economicamente, a restrição faz
sentido ao exigir que os lucros em um mercado com 4 firmas sejam menores
do que em um mercado com 3. Por outro lado, essa imposição pode ser
excessivamente restritiva já que os menores lucros das firmas que entram
no mercado depois podem advir de maiores custos conforme o número de
firmas aumenta, o que parece ser verdade dado o valor estimado para
\(\gamma_4\). Além disso, através dessa restrição, os autores tomam como
dado uma premissa central do modelo desenvolvido em Bresnahan and Reiss
(1990) e Bresnahan and Reiss (1991) que deveria ser refletida pelos
dados observados. Diante disso, decisão não parece ser justificada.

\textbf{(d)} Os resultados das estimações introduzindo \(-\alpha n\) e
\(-\alpha \log(n)\) no V ao invés das dummies está exposto nas duas
tabelas abaixo. Não há grandes diferenças em relação às estimativas
originais quando olhamos para os parâmetros \(\lambda\)'s, \(\beta\)'s e
\(\gamma\)'s, em termos qualitativos, apesar das estimativas pontuais
variarem bastante. Entretanto, agora temos apenas uma estimativa para
\(\alpha\), o que implica que o efeito da entrada das firmas no lucro é
constante para todo \(n\), na primeira especificação, ou dado por
\(-\alpha \times [\log(n) - \log(n-1)]\), na segunda, gerando um efeito
que decresce com \(n\). Isso gera estimativas bem diferentes do modelo
com dummies, especialmente quando \(n\) aumenta, pois impõe uma
estrutura mais rígida no efeito de \(n\).

\begin{table}[H]

\caption{\label{tab:unnamed-chunk-11}Tabela 4 - Bresnahan and Reiss (1991) com $-\alpha n$ substituindo dummies}
\centering
\begin{tabular}[t]{lrr}
\toprule
  & Estimativa & Erro-padrão assintótico\\
\midrule
$\lambda_1$ & -0.71 & 0.39\\
$\lambda_2$ & 2.10 & 0.97\\
$\lambda_3$ & 0.18 & 0.54\\
$\lambda_4$ & 0.45 & 0.47\\
$\beta_2$ & -0.21 & 0.58\\
$\beta_3$ & -0.02 & 0.03\\
$\beta_4$ & 0.09 & 0.03\\
$\beta_7$ & 0.01 & 0.08\\
$\alpha$ & 0.05 & 0.03\\
$\gamma_1$ & 0.50 & 0.20\\
$\gamma_2$ & 0.73 & 0.11\\
$\gamma_3$ & 0.67 & 0.12\\
$\gamma_4$ & 0.48 & 0.13\\
$\gamma_5$ & 0.22 & 0.14\\
$\gamma_L$ & -0.78 & 0.40\\
Log likelihood & -266.65 & \\
\bottomrule
\end{tabular}
\end{table}

\begin{table}[H]

\caption{\label{tab:unnamed-chunk-13}Tabela 4 - Bresnahan and Reiss (1991) com $-\alpha \log(n)$ substituindo dummies}
\centering
\begin{tabular}[t]{lrr}
\toprule
  & Estimativa & Erro-padrão assintótico\\
\midrule
$\lambda_1$ & -0.68 & 0.39\\
$\lambda_2$ & 2.19 & 0.95\\
$\lambda_3$ & 0.25 & 0.56\\
$\lambda_4$ & 0.44 & 0.48\\
$\beta_2$ & -0.18 & 0.58\\
$\beta_3$ & -0.03 & 0.03\\
$\beta_4$ & 0.09 & 0.03\\
$\beta_7$ & 0.00 & 0.07\\
$\alpha$ & 0.13 & 0.08\\
$\gamma_1$ & 0.53 & 0.21\\
$\gamma_2$ & 0.68 & 0.12\\
$\gamma_3$ & 0.65 & 0.12\\
$\gamma_4$ & 0.50 & 0.13\\
$\gamma_5$ & 0.28 & 0.13\\
$\gamma_L$ & -0.77 & 0.40\\
Log likelihood & -266.76 & \\
\bottomrule
\end{tabular}
\end{table}

Os resultados das estimações introduzindo \(\gamma n\) e
\(\gamma \log(n)\) no F ao invés das dummies está exposto nas duas
tabelas abaixo. Novamente, não há grandes diferenças em relação às
estimativas originais quando olhamos para os parâmetros \(\alpha\)'s,
\(\beta\)'s e \(\lambda\)'s, em termos qualitativos, apesar das
estimativas pontuais variarem bastante.

Analogamente aos modelos anteriores, agora temos apenas uma estimativa
para \(\gamma\), o que implica que o efeito da entrada das firmas no
custo fixo é constante para todo \(n\), na primeira especificação, ou
dado por \(\gamma \times [\log(n) - \log(n-1)]\), na segunda, gerando um
efeito que decresce com \(n\). Isso gera estimativas bem diferentes do
modelo com dummies, pois impõe uma estrutura mais rígida no efeito de
\(n\).

\begin{table}[H]

\caption{\label{tab:unnamed-chunk-15}Tabela 4 - Bresnahan and Reiss (1991) com $\gamma n$ substituindo dummies}
\centering
\begin{tabular}[t]{lrr}
\toprule
  & Estimativa & Erro-padrão assintótico\\
\midrule
$\lambda_1$ & -0.38 & 0.43\\
$\lambda_2$ & 2.32 & 0.99\\
$\lambda_3$ & 0.61 & 0.66\\
$\lambda_4$ & 0.09 & 0.45\\
$\beta_2$ & -0.38 & 0.62\\
$\beta_3$ & -0.03 & 0.03\\
$\beta_4$ & 0.00 & 0.06\\
$\beta_7$ & -0.02 & 0.08\\
$\alpha_1$ & 0.91 & 0.46\\
$\alpha_2$ & 0.13 & 0.08\\
$\alpha_3$ & 0.11 & 0.05\\
$\alpha_4$ & 0.00 & 0.03\\
$\alpha_5$ & 0.00 & 0.03\\
$\gamma$ & 0.55 & 0.06\\
$\gamma_L$ & -0.70 & 0.37\\
Log likelihood & -265.87 & \\
\bottomrule
\end{tabular}
\end{table}

\begin{table}[H]

\caption{\label{tab:unnamed-chunk-17}Tabela 4 - Bresnahan and Reiss (1991) com $\gamma \log(n)$ substituindo dummies}
\centering
\begin{tabular}[t]{lrr}
\toprule
  & Estimativa & Erro-padrão assintótico\\
\midrule
$\lambda_1$ & -0.34 & 0.47\\
$\lambda_2$ & 2.32 & 1.28\\
$\lambda_3$ & 0.38 & 0.66\\
$\lambda_4$ & 0.14 & 0.46\\
$\beta_2$ & -0.47 & 0.62\\
$\beta_3$ & -0.02 & 0.03\\
$\beta_4$ & -0.01 & 0.06\\
$\beta_7$ & -0.03 & 0.08\\
$\alpha_1$ & 0.77 & 0.46\\
$\alpha_2$ & -0.08 & 0.07\\
$\alpha_3$ & 0.09 & 0.05\\
$\alpha_4$ & 0.03 & 0.03\\
$\alpha_5$ & 0.04 & 0.03\\
$\gamma$ & 1.43 & 0.16\\
$\gamma_L$ & -0.27 & 0.34\\
Log likelihood & -267.49 & \\
\bottomrule
\end{tabular}
\end{table}

\hypertarget{questuxe3o-2}{%
\section{Questão 2}\label{questuxe3o-2}}

\textbf{(a)} Dada uma realização de \(\epsilon\), podemos mostrar que o
vetor de decisões de entrada de equilíbrio é único. Sabemos que uma
firma entra na localidade \(m\) se \(\Pi_m > 0\) ou, de forma
equivalente, \(\epsilon_m > -X_m\beta - \alpha I_{m-1}\)

Assim, se tomamos \(m=1\), temos que uma usina entra em \(1\) se: \[
\Pi_1 > 0 \Rightarrow \epsilon_1 > -X_1 \beta
\]

Como as firmas observam todo o vetor \(\epsilon\), em \(m=2\) a usina
sabe se \(I_{1} = 1\) ou não. Logo, a usina entra em \(2\) se: \[
\epsilon_2 > -X_2 \beta - \alpha\mathbbm{1}\{\epsilon_1 > -X_1 \beta\}
\]

Assim, podemos resolver esse problema de forma sequencial, computando a
escolha em \(m=1\), depois usar \(I_1\) para computar a escolha em
\(m=2\) e assim por diante para todas as \(M\) localidades, o que mostra
que o vetor de decisões de entrada de equilíbrio é único já que a
decisão depende apenas da realização do próprio \(\epsilon\) após a
escolha de \(m-1\) e essa escolha não é afetada pela decisão em \(m\).
Nesse caso, o sinal de \(\alpha\) pode alterar o vetor de decisões de
entrada, pois um \(\alpha\) negativo pode reduzir a probabilidade de se
instalar na localidade, ao contrário do \(\alpha\) positivo, mas não
afeta a unicidade do equilíbrio.

Sabendo a distribuição de \(\epsilon\), podemos estimar esse equilíbrio
de forma sequencial. Seja \(P_m\) a probabilidade de entrada em \(m\),
podemos então construir a verossimilhança com base nas probabilidades:
\[
P_1 = P(\epsilon_1 > -X_1 \beta) = 1 - F(-X_1 \beta)
\] \begin{align*}
    P_2 &= P(\epsilon_2 > -X_2 \beta - \alpha I_1) \\        
  &= 1 - F(-X_2 \beta - \alpha I_1)
\end{align*} \[
\vdots
\] \begin{align*}
    P_M &= P(\epsilon_M > -X_M \beta - \alpha I_{M-1}) \\        
  &= 1 - F(-X_M \beta - \alpha I_{M-1})
\end{align*}

onde \(\epsilon \sim F\).

Como os \(I_m\) são variáveis binárias, a maximização da verossimilhança
toma a forma: \[
\max \prod_{m=1}^{M} P_{m}^{I_{m}}\left(1-P_{m}\right)^{1-I_{m}}
\]

\textbf{(b)} Agora que cada firma observa apenas o \(\epsilon_m\) de sua
localidade, a usina entra em \(m\) se: \[
\mathbb{E}[\Pi_m] > 0 \Rightarrow \epsilon_m > -X_m \beta - \alpha P_{m-1}
\] já que \(I_{m-1}\) é binária.

Podemos então construir a verossimilhança com base nas probabilidades:
\[
P_1 = P(\epsilon_1 > -X_1 \beta) = 1 - F(-X_1 \beta)
\] \begin{align*}
    P_2 &= P(\epsilon_2 > -X_2 \beta - \alpha P_1) \\        
  &= 1 - F(-X_2 \beta - \alpha P_1)
\end{align*} \[
\vdots
\] \begin{align*}
    P_M &= P(\epsilon_M > -X_M \beta - \alpha P_{M-1}) \\        
  &= 1 - F(-X_M \beta - \alpha P_{M-1})
\end{align*}

Assim, o equilíbrio ainda é único, podendo novamente ser resolvido de
forma sequencial como no item (a). Como os \(I_m\) são variáveis
binárias, a maximização da verossimilhança toma a forma: \[
\max \prod_{m=1}^{M} P_{m}^{I_{m}}\left(1-P_{m}\right)^{1-I_{m}}
\]

\textbf{(c)} Nessa versão, o \(I\) de equilíbrio não é mais único.
Podemos ilustrar os casos possíveis para \(M=2\).

Quando \(\gamma > 0\) e \(\alpha > 0\), múltiplos equilíbrios aparecem
no cenário onde \(-X_1 \beta - \gamma < \epsilon_1 < -X_1 \beta\) e
\(-X_2 \beta - \alpha < \epsilon_2 < -X_2 \beta\). Nesse caso, temos
dois possíveis equilíbrios: (1, 1) e (0, 0).

Quando \(\gamma < 0\) e \(\alpha < 0\), múltiplos equilíbrios aparecem
no cenário onde \(-X_1 \beta - \gamma > \epsilon_1 > -X_1 \beta\) e
\(-X_2 \beta - \alpha > \epsilon_2 > -X_2 \beta\). Nesse caso, temos
dois possíveis equilíbrios: (1, 0) e (0, 1).

Os casos onde \(\gamma > 0\) e \(\alpha < 0\) ou \(\gamma < 0\) e
\(\alpha > 0\) são análogos e não há equilíbrio em estratégias puras no
caso onde temos \(\epsilon_1\) e \(\epsilon_2\) na zona intermediária
entre \([-X_1 \beta - \gamma, -X_1 \beta]\) e
\([-X_2 \beta - \alpha, -X_2 \beta]\).

Nesse caso, não podemos estimar a verossimilhança como nos casos
anteriores devido à multiplicidade de equilíbrios. A solução é ou
selecionar arbitrariamente um dos equilíbrios que surgem ou modelar o
jogo como não-simultâneo ou definir uma verossimilhança composta. A
opção de jogo não-simultâneo pode fazer sentido caso uma das usinas
envolvidas no equilíbrio múltiplo seja mais antiga do que a outra e já
esteja estabelecida, o que sugere uma ordem para as ações dos jogadores.

\hypertarget{questuxe3o-3}{%
\section{Questão 3}\label{questuxe3o-3}}

\textbf{(a)} Como só temos dados para um período, precisamos interpretar
os dados disponíveis da mesma forma que Bresnahan and Reiss (1991), de
que os mercados observados são um equilíbrio de longo prazo. Vamos
estudar a verossimilhança para entender o que é possível de ser
estimado. A probabilidade de observar mercados sem firmas em \(i\) é
dada por: \begin{align*}
    P(0 \text{ firmas}) &= P(\Pi_i(n=1) < 0)  \\        
  &= P(X_i\beta + h(1) - \Delta + \epsilon_i < 0) \\
  &= 1 - F(\Delta - h(1) - X_i\beta)
\end{align*}

onde \(\epsilon \sim F\).

A probabilidade de observar mercados com \(k\) firmas em \(i\), com
\(k\) entre 1 e \(N-1\) é dada por: \begin{align*}
    P(k \text{ firmas}) &= P(\Pi_i(n=k) \geq -\Sigma \text{ e } \Pi_i(n=k+1) < 0) \\        
  &= F(-\Sigma - h(k) - X_i\beta) - F(\Delta - h(k+1) - X_i\beta)
\end{align*}

Por fim, a probabilidade de observar mercados com \(N\) firmas em \(i\)
é: \begin{align*}
    P(N \text{ firmas}) &= P(\Pi_i(n=N) \geq -\Sigma) \\        
  &= F(-\Sigma - h(N) - X_i\beta)
\end{align*}

Como as funções \(h\) sempre aparecem no ``intercepto'' conjuntamente
com \(\Delta\) ou \(\Sigma\), não é possível estimar os parâmetros
separadamente. Isso ocorre pois existem infinitas combinações possíveis
para essas somas. Por exemplo, podemos ter um maior valor de \(\Delta\)
sem alteração no valor dos interceptos se modificarmos também as
estimativas de \(h\) e \(-\Sigma\), não sendo possível identificá-los
unicamente.

\textbf{(b)} Dada a limitação da base de dados, não levaremos em conta a
identidade das firmas presentes no mercado. O que importa será apenas a
variação do número de firmas de um período para o outro. Observando o
caso de dois períodos, temos 3 casos possíveis:

\begin{itemize}
\item
  \(n_2 > n_1\): a entrada de firmas ocorre quando a \(n_2\)-ésima
  entrante tem lucros não negativos e a entrante seguinte tem lucros
  negativos. \[
  X_{i2}\beta + h(n_2) + \epsilon_{i2} \geq \Delta
  \] \[
  X_{i2}\beta + h(n_2 + 1) + \epsilon_{i2} < \Delta
  \]
\item
  \(n_2 < n_1\): a saída de firmas ocorre quando não é mais lucrativo
  para \(n_1 - n_2\) incumbentes continuarem no mercado. \[
  X_{i2}\beta + h(n_2) + \epsilon_{i2} \geq -\Sigma
  \] \[
  X_{i2}\beta + h(n_2 + 1) + \epsilon_{i2} < -\Sigma
  \]
\item
  \(n_2 = n_1\): o número de firmas não se altera quando não é lucrativa
  a entrada de uma nova firma mas a permanência das incumbentes é. \[
  X_{i2}\beta + h(n_2) + \epsilon_{i2} \geq -\Sigma
  \] \[
  X_{i2}\beta + h(n_2 + 1) + \epsilon_{i2} < \Delta
  \]
\end{itemize}

Assim, podemos definir as probabilidades de observar um mercado com
\(k\) firmas em \(t=1\) e \(n\) firmas em \(t=2\): \[
P(n_2 = n, n_1 = k)=\left\{\begin{array}{cc}
F(\Delta - h(n + 1) - X_{i2}\beta) - F(\Delta - h(n) - X_{i2}\beta) & n > k \\
F(\Delta - h(n + 1) - X_{i2}\beta) - F(-\Sigma - h(n) - X_{i2}\beta) & n = k \\
F(-\Sigma - h(n + 1) - X_{i2}\beta) - F(-\Sigma - h(n) - X_{i2}\beta) & n < k
\end{array}\right.
\]

Novamente, não é possível estimar os parâmetros separadamente. Isso
ocorre pois existem infinitas combinações possíveis para essas somas.
Por exemplo, podemos ter um maior valor de \(\Delta\) sem alteração no
valor dos interceptos se modificarmos também as estimativas de \(h\) e
\(-\Sigma\), não sendo possível identificá-los unicamente.

\textbf{(c)} Se as firmas são forward-looking, o lucro esperado
descontado da firma que entra na localidade \(i\) em \(t=\tau\) e sai em
\(T\) é dado por: \[
E\left[\sum_{t=\tau}^{T-1} \delta^{t-\tau}[X_{it}\beta + h(n_t) + \epsilon_{it}] - \Delta - \delta^{T}\Sigma \right]
\] onde \(\delta\) é um fator de desconto.

Assim, a firma entra se o lucro esperado descontado dela é maior ou
igual a zero. Isso ocorre porque existem várias potenciais entrantes, o
que não permite que a firma atrase sua decisão de entrada para conseguir
lucros maiores no futuro, já que outras firmas entrarão em seu lugar.

Observando mais especificamente o caso de dois períodos, temos novamente
3 casos possíveis:

\begin{itemize}
\item
  \(n_2 > n_1\): a entrada de firmas ocorre quando a \(n_2\)-ésima
  entrante tem lucros não negativos e a entrante seguinte tem lucros
  negativos. \[
  X_{i2}\beta + h(n_2) + \epsilon_{i2} + E\left[\sum_{t=3}^{T-1} \delta^{t-\tau}[X_{it}\beta + h(n_t) + \epsilon_{it}] \right] \geq \Delta
  \] \[
  X_{i2}\beta + h(n_2+1) + \epsilon_{i2} + E\left[\sum_{t=3}^{T-1} \delta^{t-\tau}[X_{it}\beta + h(n_t) + \epsilon_{it}] \right] < \Delta
  \]
\item
  \(n_2 < n_1\): a saída de firmas ocorre quando não é mais lucrativo
  para \(n_1 - n_2\) incumbentes continuarem no mercado. \[
  X_{i2}\beta + h(n_2) + \epsilon_{i2} + E\left[\sum_{t=3}^{T-1} \delta^{t-\tau}[X_{it}\beta + h(n_t) + \epsilon_{it}] \right] \geq -\Sigma
  \] \[
  X_{i2}\beta + h(n_2+1) + \epsilon_{i2} + E\left[\sum_{t=3}^{T-1} \delta^{t-\tau}[X_{it}\beta + h(n_t) + \epsilon_{it}] \right] < -\Sigma
  \]
\item
  \(n_2 = n_1\): o número de firmas não se altera quando não é lucrativa
  a entrada de uma nova firma mas a permanência das incumbentes é. \[
  X_{i2}\beta + h(n_2) + \epsilon_{i2} + E\left[\sum_{t=3}^{T-1} \delta^{t-\tau}[X_{it}\beta + h(n_t) + \epsilon_{it}] \right] \geq -\Sigma
  \] \[
  X_{i2}\beta + h(n_2+1) + \epsilon_{i2} + E\left[\sum_{t=3}^{T-1} \delta^{t-\tau}[X_{it}\beta + h(n_t) + \epsilon_{it}] \right] < \Delta
  \]
\end{itemize}

Tendo apenas dois períodos disponíveis, não é possível estimar algo
nesse sentido. Em Bresnahan and Reiss (1994), os autores utilizam o
lucro de um período seguinte (\(t=3\)) para representar todo o fluxo
futuro de lucros, entretanto, aqui temos apenas dois períodos.

Se fizermos a suposição de que o lucro em \(t=2\) representa todo o
fluxo futuro cairíamos no mesmo problema do item anterior. Novamente,
não é possível estimar os parâmetros \(\Delta\), \(h\) e \(-\Sigma\)
separadamente.

\vspace{1cm}

\hypertarget{referuxeancias}{%
\section{Referências}\label{referuxeancias}}

\begin{itemize}
\tightlist
\item
  Bresnahan, T. F., \& Reiss, P. C. (1990). Entry in monopoly market.
  The Review of Economic Studies, 57(4), 531-553.
\item
  Bresnahan, T. F., \& Reiss, P. C. (1991). Entry and competition in
  concentrated markets. Journal of Political Economy, 99(5), 977-1009.
\item
  Bresnahan, T. F., \& Reiss, P. C. (1994). Measuring the importance of
  sunk costs. Annales d'Economie et de Statistique, 181-217.
\end{itemize}

\end{document}
